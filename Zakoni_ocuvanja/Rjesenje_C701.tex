

\begin{enumerate}[label=\alph*)]
 \item Ako je brzina stalna tada je rezultantna sila na automobil jednaka je nuli; 
$\vec{v}=konstanta    \ \ \Rightarrow \ \ \vec{F}_R=\vec{0}.$

$$ \vec{F} + \vec{F}_{otp} + \vec{G}_{||} + \vec{G}_{\bot} + \vec{R} = \vec{0} \ \ \ \  /\cdot\vec{j} $$
$$ F - F_{otp} - mg\sin\vartheta = 0$$
$$ F = F_{otp} + mg\sin\vartheta  $$
$$ F = 2000N + 2000kg\ 9,81ms^{-2}\ \sin15^\circ = 7078,03\ N $$
 \item  $$ W=\vec{F}\Delta \vec{r}=F\Delta r\cos 0^\circ $$
 Pomak automobila možemo izraziti preko visine kosine i kuta
 $$ W = F\frac{h}{\sin\vartheta} = 7078,03\frac{60m}{\sin15^\circ} = 1640844\ J $$
 \item $$ P = \vec{F}\cdot\vec{v}=Fv $$
Iznos brzine automobila je $ v = 60\ kmh^{-1} = 60\frac{1000\ m}{3600\ s}=16,67\ ms^{-1}$
$$P=7078,03\ N 16,67\ ms^{-1} = 117967\ W$$
\end{enumerate}
