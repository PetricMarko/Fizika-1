

Moment tromosti sustava $I$ je zbroj momenta tromosti svake kugel, $I=2I_{kugla}$
Kako bismo odredili moment tromosti kugle koristimo teorem o paralelnim osima (Steinerov teorem):
$$
I_{kugla}= I_T+Md^2
$$

$$
I_{kugla}=\frac{2}{5}MR^2+M(\frac{L}{2}+R)^2
$$
gdje je $M$ masa jedne kugle, $R$ je njezin radijus, a $L$ je udaljenost između kugli. Udaljenost osi rotacije od centra mase kugle je $d=\frac{L}{2}+R$. Izrazimo masu pomoću gustoće i volumena kugle ($V=\frac{4}{3}R^3\pi$) i dobivamo moment tromosti jedne kugle:
$$
I_{kugle}=\frac{4}{3}\pi\rho R^3\left[\frac{2}{5}R^2 + \left(\frac{L}{2}+R \right)^2\right].
$$
Moment tromosti sustava je:
$$
I=2I_{kugle}=\frac{8}{3}\pi2700\ kgm^{-3} (0,04\ m)^3\left[\frac{2}{5}(0,04\ m)^2 + \left(\frac{0,1\ m}{2}+(0,04\ m) \right)^2\right]
$$

$$ I=0,01265\ kgm^2.$$

