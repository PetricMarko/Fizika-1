

\begin{enumerate}[label=\alph*)]
\item Na tijelo A djeluju sila teže ($\vec{G}_A$) prema dolje koju rastavljamo na dvije komponente:
silu okomitu na kosinu ($ \vec{G}_{A,\bot}$) i silu usporednu s kosinom prema dolje ($\vec{G}_{A,||} $), zatim djeluje sila 
trenja ($\vec{F}_{tr,A} $), sila reakcije podloge $\vec{R}_A$ i sila kojom uteg B vuče uteg A (sila napetosti niti $\vec{T}_{BA} $).  
Na uteg B djeluju samo dvije sile, sila teža prema dolje ($\vec{G}_B$) i napetost niti prema gore ($\vec{T}_{AB}$).

Sila napetosti niti kojom djeluje uteg A na uteg B jednaka je po iznosu sili napetosti kojom uteg B djeluje na uteg A stoga pišemo
$$ |\vec{T}_{AB}|=|\vec{T}_{BA}|=T. $$ 
\item 
Za uteg B možemo pisati
$$\vec{G}_B+\vec{T}_{AB}=m_B\vec{a},$$
\begin{equation}
 G_B-T=m_Ba   \ \ \Rightarrow \ \  T=m_B(g-a).  
 \label{zadatak_5_2_1}
\end{equation}
Zapisujemo sve sile koje djeluju na uteg A
$$ \vec{G}_{A,||}+\vec{G}_{A,\bot}+\vec{R}_A+\vec{F}_{tr,A} + \vec{T}_{BA}=m_A\vec{a}.  $$
Radimo projekciju sila na smjer gibanja
$$ G_{A,||} -F_{tr,A}+T=m_A a  $$
$$ m_Ag\sin\alpha-\mu_k m_Ag\cos\alpha +T = m_Aa $$
Napetost niti možemo zamjeniti izrazom \ref{zadatak_5_2_1} i dobivamo
$$ m_Ag\sin\alpha-\mu_k m_Ag\cos\alpha +m_Bg-m_Ba = m_Aa. $$ 
Nakom sređivanja dobivamo konačni izraz
$$ (m_A\sin\alpha-\mu_k m_A\cos\alpha +m_B)g =( m_A+m_B)a $$
$$a=\frac{m_A(\sin\alpha-\mu_k \cos\alpha) +m_B}{ m_A+m_B}\ g. $$
Uvrstimo zadane vrijednosti
$$a=\frac{10\ kg(\sin30^\circ-0,2\cos30^\circ)}{10\ kg+5\ kg}\ 9,81\ ms^{-2}=5,41 \ ms^{-2}$$

 \item Kako bismo dobili iznos sile napetosti niti uvrštavamo dobivenu akceleraciju u izrac \ref{zadatak_5_2_1}
 $$ T=5\ kg(9,81 \ ms^{-2}- 5,41 \ ms^{-2})=22\ N $$
 
 
\end{enumerate}


